%% LaTeX2e class for student theses
%% thesis.tex
%% 
%% Karlsruhe Institute of Technology
%% Institute for Program Structures and Data Organization
%% Chair for Software Design and Quality (SDQ)
%%
%% Dr.-Ing. Erik Burger
%% burger@kit.edu
%%
%% Version 1.3.3, 2018-04-17

%% Available page modes: oneside, twoside
%% Available languages: english, ngerman
%% Available modes: draft, final (see README)
\documentclass[twoside, english, final]{sdqthesis}

%% ---------------------------------
%% | Information about the thesis  |
%% ---------------------------------

%% Name of the author
\author{Lucas Werkmeister}

%% Title (and possibly subtitle) of the thesis
\title{Schema Inference on Wikidata}

%% Type of the thesis 
\thesistype{Master's Thesis}

%% Change the institute here, ``IPD'' is default
% \myinstitute{Institute for \dots}

%% You can put a logo in the ``logos'' directory and include it here
%% instead of the SDQ logo
% \grouplogo{myfile}
%% Alternatively, you can disable the group logo
% \nogrouplogo

%% The reviewers are the professors that grade your thesis
\reviewerone{Prof. Reussner}
\reviewertwo{Prof. Sack}

%% The advisors are PhDs or Postdocs
\advisorone{Dr. Koutraki}

%% Please enter the start end end time of your thesis
\editingtime{16. April 2018}{15. October 2018}

\settitle

%% --------------------------------
%% | Settings for word separation |
%% --------------------------------

%% Describe separation hints here.
%% For more details, see 
%% http://en.wikibooks.org/wiki/LaTeX/Text_Formatting#Hyphenation
\hyphenation{
% me-ta-mo-del
}

%% --------------------------------
%% | Bibliography                 |
%% --------------------------------

%% Use biber instead of BibTeX, see README
\usepackage[citestyle=numeric,style=numeric,backend=biber]{biblatex}
\addbibresource{thesis.bib}

%% --------------------------------
%% | Glossary                     |
%% --------------------------------

\usepackage{glossaries}
\makeglossaries
\newacronym{shex}{ShEx}{Shape Expressions}
\newacronym{w3c}{W3C}{World Wide Web Consortium}


%% --------------------------------
%% | Listings                     |
%% --------------------------------

% https://tex.stackexchange.com/a/279245/25264
\usepackage{float}
\usepackage{subcaption}
\usepackage{listings}
\usepackage{cleveref}
\newfloat{lstfloat}{htbp}{lop} % TODO I’m not sure about that placement, perhaps just tbp (top, bottom, own page)?
\floatname{lstfloat}{Listing}
\crefalias{lstfloat}{listing}
\lstset{
  basicstyle=\ttfamily,
}

%% --------------------------------
%% | Pseudo-code                  |
%% --------------------------------
\usepackage{algpseudocode}
\usepackage{algorithm}

%% --------------------------------
%% | TikZ                         |
%% --------------------------------

\usepackage{tikz}
\usetikzlibrary{shapes,shapes.multipart,arrows,positioning}

%% --------------------------------
%% | Miscellaneous packages       |
%% --------------------------------

\usepackage[binary-units]{siunitx}
\usepackage[section,above,below]{placeins}

%% --------------------------------
%% | Macros                       |
%% --------------------------------

\newcommand{\Q}[1]{\href{http://www.wikidata.org/entity/#1}{#1}}
\newcommand{\QL}[2]{\href{http://www.wikidata.org/entity/#1}{“#2” (#1)}}
\renewcommand{\P}[1]{\href{http://www.wikidata.org/entity/#1}{#1}}
\newcommand{\PL}[2]{\href{http://www.wikidata.org/entity/#1}{“#2” (#1)}}
\newcommand{\prefix}[1]{\texttt{#1}}
\newcommand{\PName}[1]{\texttt{#1}}

\newcommand{\filename}[1]{\texttt{#1}}
\newcommand{\dirname}[1]{\texttt{#1/}}
\newcommand{\command}[1]{\texttt{#1}}
\newcommand{\variable}[1]{\textit{#1}}
\newcommand{\branchname}[1]{\texttt{#1}}

\newcommand{\loc}[1]{\href{http://id.loc.gov/authorities/names/#1}{#1}}
\newcommand{\viaf}[1]{\href{https://viaf.org/viaf/#1}{#1}}
\newcommand{\imdbName}[1]{\href{https://www.imdb.com/name/#1}{#1}}
\newcommand{\TwitterAccount}[1]{\href{https://twitter.com/#1}{@#1}}

\newcommand{\Instruction}[1]{\text{\textit{#1}}}
\newcommand{\Variable}[1]{\textup{#1}}
\newcommand{\FunctionName}[1]{\textproc{#1}}

\newcommand{\wdsiJob}[1]{\href{https://tools.wmflabs.org/wd-shex-infer/job/#1}{job \##1}}

% \minsec is always used in a context where it is clear that m means minutes and not metres
\DeclareSIUnit{\shortMinute}{m}
\newcommand{\minsec}[2]{\SI{#1}{\shortMinute}~\SI{#2}{\second}}

%% ====================================
%% ====================================
%% ||                                ||
%% || Beginning of the main document ||
%% ||                                ||
%% ====================================
%% ====================================
\begin{document}

%% Set PDF metadata
\setpdf

%% Set the title
\maketitle

%% The Preamble begins here
\frontmatter

%% LaTeX2e class for student theses: Declaration of independent work
%% sections/declaration.tex
%% 
%% Karlsruhe Institute of Technology
%% Institute for Program Structures and Data Organization
%% Chair for Software Design and Quality (SDQ)
%%
%% Dr.-Ing. Erik Burger
%% burger@kit.edu
%%
%% Version 1.3.3, 2018-04-17

\thispagestyle{empty}
\null\vfill
\noindent\hbox to \textwidth{\hrulefill} 
\iflanguage{english}{I declare that I have developed and written the enclosed
thesis completely by myself, and have not used sources or means without
declaration in the text.}%
{Ich versichere wahrheitsgemäß, die Arbeit
selbstständig angefertigt, alle benutzten Hilfsmittel vollständig und genau
angegeben und alles kenntlich gemacht zu haben, was aus Arbeiten anderer
unverändert oder mit Änderungen entnommen wurde.}
 
 
%% ---------------------------------------------
%% | Replace PLACE and DATE with actual values |
%% ---------------------------------------------
\textbf{PLACE, DATE}
\vspace{1.5cm}
 
\dotfill\hspace*{8.0cm}\\
\hspace*{2cm}(\theauthor) 
\cleardoublepage

\setcounter{page}{1}
\pagenumbering{roman}

%% ----------------
%% |   Abstract   |
%% ----------------
 
%% For theses written in English, an abstract both in English
%% and German is mandatory.
%%
%% For theses written in German, a German abstract is sufficient.
%%
%% The text is included from the following files:
%% - sections/abstract

\includeabstract

%% ------------------------
%% |   Table of Contents  |
%% ------------------------
\tableofcontents

\listoffigures
\listoftables

%% -----------------
%% |   Main part   |
%% -----------------

\mainmatter

%% LaTeX2e class for student theses
%% sections/content.tex
%% 
%% Karlsruhe Institute of Technology
%% Institute for Program Structures and Data Organization
%% Chair for Software Design and Quality (SDQ)
%%
%% Dr.-Ing. Erik Burger
%% burger@kit.edu
%%
%% Version 1.3.3, 2018-04-17

\chapter{Introduction}
\label{ch:Introduction}

As Wikidata,
the free knowledge base in the Wikimedia movement,
continues to grow in volume and scope and is used by more and more other parties, % TODO “other parties” is weird; TODO citation for growth?
its data quality has been identified as one of the most important areas of development in the future \cite{wdcon2017-sotp}: % TODO \cite before colon looks ugly
in order for Wikidata to be useful,
its data must be trustworthy and available in a consistent format.
Unchecked vandalism discourages data reuse,
while inconsistent data models make it significantly more difficult or even impossible.

To combat these problems,
several quality control mechanisms are used on Wikidata.
Recently, editors have begun exploring the use of \acrlong{shex}
as another quality control mechanism to use.
Compared to the more established, Wikidata-specific quality constraints system,
\acrlong{shex} are more powerful and expressive,
and are also not specific to Wikidata alone.
However, schemas for \acrlong{shex} are tedious to write by hand.

Automatically inferring schemas from Wikidata items
promises to simplify the schema authoring process:
instead of manually putting together the schema,
describing shapes for different classes of items,
one simply selects a set of items,
and a schema is automatically generated based on the data about these items.
If the selected items have been carefully edited
to conform to a pre-existing schema,
perhaps described informally or only present in the minds of the editors,
then the result may be a formalization of that schema;
alternatively, applying the same process to a less curated set of input items
may result in a concise summary of the current schema of those items % TODO “concise” is not really the right word…
and possibly even demonstrate problems in the input data.

% TODO probably mention WikiProject ShEx?

This thesis investigates the usefulness and applicability
of automatically inferring schemas for Wikidata from sets of exemplary items.
It builds on the existing RDF2Graph \cite{vanDam2015} program,
updating and adapting it to support Wikidata
and automating the whole inference process.
This was then made available to the whole Wikidata community % TODO kinda mixing tenses here
by incorporating it into a web-based tool.

The rest of the thesis % TODO a) “the rest” or just “this thesis”? b) “this thesis”? “this work”?
is organized as follows.
\Cref{ch:Background} explains concepts that are required
to understand the rest of the work. % TODO the rest of the thesis?
\Cref{ch:RDF2Graph+Wikidata} describes general updates for RDF2Graph
as well as changes that were made to add Wikidata support to it.
\Cref{ch:wdsi} introduces the Wikidata Shape Expressions Inference Tool
and describes its design and implementation.
\Cref{ch:Evaluation} then evaluates the usefulness of the tool and the resulting schemas.
Finally, \cref{ch:Conclusion} summarizes the results and concludes the thesis.

%% LaTeX2e class for student theses
%% sections/content.tex
%% 
%% Karlsruhe Institute of Technology
%% Institute for Program Structures and Data Organization
%% Chair for Software Design and Quality (SDQ)
%%
%% Dr.-Ing. Erik Burger
%% burger@kit.edu
%%
%% Version 1.3.3, 2018-04-17

\chapter{Background}
\label{ch:Background}


\section{Wikidata}
\label{ch:Background:Wikidata}

Explain the Wikidata data model,
the central role of the community,
and perhaps some fundamental properties.

Also explain the RDF export
and the query service,
and how they’re derived from Wikidata,
not the canonical source of truth.


\section{Shape Expressions}
\label{ch:Background:ShEx}

How they work.

Difference between v1 and v2?


\section{RDF2Graph}
\label{ch:Background:RDF2Graph}

Summarize how RDF2Graph works?
The distinction between the steps
(main, simplify, ShEx export)
is likely to be significant for the rest of the thesis.

%% LaTeX2e class for student theses
%% sections/content.tex
%% 
%% Karlsruhe Institute of Technology
%% Institute for Program Structures and Data Organization
%% Chair for Software Design and Quality (SDQ)
%%
%% Dr.-Ing. Erik Burger
%% burger@kit.edu
%%
%% Version 1.3.3, 2018-04-17

% TODO find a better title for this chapter
\chapter{Applying RDF2Graph to Wikidata}
\label{ch:RDF2Graph+Wikidata}

% TODO find a better title for this section
\section{General RDF2Graph Updates}
\label{sec:RDF2Graph+Wikidata:updates}

% TODO find a better title for this section
\section{Wikidata Support}
\label{sec:RDF2Graph+Wikidata:Wikidata}

The most fundamental change to make RDF2Graph support Wikidata
was % TODO is?
to change the type predicates used.
RDF2Graph heavily relies on the type information of the RDF graph it inspects,
assuming a one-to-one mapping between classes and shapes,
and it uses the standard predicates \PName{rdf:type} and \PName{rdfs:subClassOf}
to determine the class of a subject and the superclasses of a class, respectively.
However, Wikidata does not use these standard predicates:
The “instance of” and “subclass of” information of an item
are regular statements like any other Wikidata statement, % TODO the information are? hm
using the properties \P{P31} and \P{P279}, respectively.
In the \gls{rdf} export, \PName{rdf:type} and \PName{rdfs:subClassOf} are only used
as part of the meta-model, % TODO meta-model? hm
assigning each item the class \PName{wikibase:Item}, a subclass of \PName{wikibase:Entity}.
To use the type information within the data instead of this meta-model,
RDF2Graph was changed to use the predicates \PName{wdt:P31} and \PName{wdt:P279} % TODO we haven’t seen the wdt: syntax yet, should that be in the Background?
instead of \PName{rdf:type} and \PName{rdfs:subClassOf}.

Another important aspect is that it’s not feasible to run RDF2Graph on the entirety of Wikidata at once.
For example, the 2018-08-20 full Wikidata dump is \SI{42.7}{\giga\byte} large after gzip compression % TODO >300GB uncompressed (309657344438 bytes), which is probably more impressive
and \num{8316156305} lines long uncompressed,
with most lines corresponding to one triple (except for some blank lines).
More importantly, inferring a single schema from all of Wikidata is also not the intention of this work:
the intention is to infer a schema from a particular set of exemplary items, % TODO awkward passive voice (“*the* intention” is *my* intention, I guess)
and ignore similar items which are not as exemplary (e.~g. not as well maintained)
as well as unrelated items. % TODO that should be explained in more detail in the Introduction, which I haven’t written yet as of this writing
So instead of running RDR2Graph directly against the SPARQL endpoint of the Wikidata Query Service,
a process was set up % TODO super awkward passive voice
to download the data of all selected items as well as the items they directly link to,
load that into a single N-Triples file,
serve that file via a local Fuseki SPARQL server,
run RDF2Graph against that server,
and finally run the ShEx exporter against RDF2Graph’s results.
To make it easier to run, the entire process is controlled by a Makefile in the RDF2Graph repository,
so that after creating a \filename{example.entities.sparql} file with a SPARQL query selecting the exemplary items,
it is sufficient to run \command{make example.shex} to run the entire process and generate the ShEx file.

% TODO find a better title for this section
\section{Schema Reduction}
\label{sec:RDF2Graph+Wikidata:schema-reduction}


% TODO find a better title for this chapter
\chapter{The Wikidata ShEx Inference Tool}
\label{ch:wdsi}

% TODO find a better title for this section
\section{Toolforge Support}
\label{sec:wdsi:Toolforge}

% TODO find a better title for this section
\section{Abstract Considerations}
\label{sec:wdsi:abstract}

% TODO find a better title for this section
\section{Utilities}
\label{sec:wdsi:utilities}

%% LaTeX2e class for student theses
%% sections/evaluation.tex
%% 
%% Karlsruhe Institute of Technology
%% Institute for Program Structures and Data Organization
%% Chair for Software Design and Quality (SDQ)
%%
%% Dr.-Ing. Erik Burger
%% burger@kit.edu
%%
%% Version 1.3.3, 2018-04-17

\chapter{Evaluation}
\label{ch:Evaluation}

% TODO find a better title for this section
\section{Schema Quality}
\label{sec:Evaluation:quality}

In general, the quality of the inferred schemas is satisfactory.
On occasion, parts of the schema may not make sense –
for example, if a shape declares that the value for a \PL{P735}{given name} statement
should be a \QL{Q202444}{given name} or a \QL{Q101352}{family name} –
but these problems are usually not the fault of the inference process,
but rather point to problems in the input data:
in this example, someone probably used an item for a family name
instead of the identical given name. % TODO example!
% TODO more about the schema quality (with examples!)
% TODO elaborate

% TODO find a better title for this section
\section{Duration of the inference process}
\label{sec:Evaluation:duration}

While attempts to validate data against the inferred schema
suffer due to the size of the input data and the schemas,
no such problems appear to plague the inference process,
which, while slow, has a more reliable runtime.
The various jobs that were run on the Wikidata Shape Expressions Inference Tool,
whose execution times range from five or ten minutes to several hours,
generally show a linear relationship between the number of triples in the input data
and the total duration of the inference job.
It is expected that this linear relationship will continue to hold for larger input data sets,
and that there is no fundamental limit barring the inference of ever larger data sets,
provided one is willing to wait long enough for the results.
% TODO add the data – here or in an appendix?

It should be noted % TODO awkward passive voice
that this is a relation between the number of triples and the execution time,
but the triples are not directly the input of the inference process:
that input is a SPARQL query which selects a set of input \emph{entities},
and the triples are the statements of those entities and the entities they link to,
downloaded in an early step of the inference process.
The relationship between the \emph{number of entities} selected and the execution time
is much more chaotic, % TODO add the data for this as well?
since it highly depends on the selected entities:
how many statements they themselves have,
as well as how many other entities they link to.
However, the data download step is itself a fairly cheap % TODO “cheap” acceptable?
step of the inference process,
so it is not very difficult to derive the number of triples,
from which it is then possible to predict the total execution type of the inference process fairly reliably.

This suggests two possible future improvements
for the Wikidata Shape Expressions Inference Tool:
to report to the user how long a job is expected to take,
as soon as the download step has finished;
and to reject jobs which resulted in too much data,
and are expected to take far too long to be tolerable.
Currently, the tool merely suggests to its users
that their queries should not select more than about fifty entities,
but does not implement any kind of hard limit beyond that.

% TODO user feedback for the tool

% TODO unable to validate directly

% TODO useful as a basis for manually assembled schemas (which can then be used to validate)

%% LaTeX2e class for student theses
%% sections/conclusion.tex
%% 
%% Karlsruhe Institute of Technology
%% Institute for Program Structures and Data Organization
%% Chair for Software Design and Quality (SDQ)
%%
%% Dr.-Ing. Erik Burger
%% burger@kit.edu
%%
%% Version 1.3.3, 2018-04-17

\chapter{Conclusion}
\label{ch:Conclusion}

The goal of this thesis was to investigate how \gls{shex} schemas can be automatically inferred for Wikidata,
and how useful the resulting schemas are.
This was done using an updated and adapted version of the RDF2Graph software,
which was made available to the Wikidata community
through a new web-based tool.

In addition to the changes specific to Wikidata,
many general improvements to RDF2Graph were made over the course of this thesis,
making it easier to use and more robust on any graph.
All these changes are available under the same free software license as the original RDF2Graph,
and I hope that some of them will be included in the original source code repository in the future.

When attempting to validate other items against the inferred schemas,
an unexpected problem arose:
none of the existing \gls{shex} validators were able to reliably perform the validation without crashing.
Several strategies were attempted to remediate this,
both in the schema extraction and in the validators,
but this was ultimately unsuccessful.

However, this does not mean the schemas are not useful.
Sometimes, problems in the input data can manifest themselves
in the form of unusual predicates or target classes in a schema,
which an attentive reader can notice and trace back to the problematic items in the input.
And the full, automatically inferred schemas
can also form the basis for shorter schemas manually extracted from the longer ones,
which can either be validated directly
(now without problems from the validators)
or be further refined by data model experts, % TODO “data model experts” sounds weird
making the automatically inferred schemas a useful basis for manually curated schemas.

% TODO next steps / future work
% * operate on full statements
% * some next steps for tool, see tool chapter
% * rewrite rest of ShEx export – we spend half of our wall-clock time (and even more of our CPU time?) there, doing what feels like not very much to the JSON
% * find other ways to make validation work
% * further improve simplification
%   * when to merge classes
%   * throw away rarely used classes?

% TODO finish off conclusion?
% TODO conclusion should probably be longer


%% --------------------
%% |   Bibliography   |
%% --------------------

%% Add entry to the table of contents for the bibliography
\printbibliography[heading=bibintoc]

%% --------------------
%% |   Glossary       |
%% --------------------

\printglossaries

%% ----------------
%% |   Appendix   |
%% ----------------
\appendix
%% LaTeX2e class for student theses
%% sections/apendix.tex
%% 
%% Karlsruhe Institute of Technology
%% Institute for Program Structures and Data Organization
%% Chair for Software Design and Quality (SDQ)
%%
%% Dr.-Ing. Erik Burger
%% burger@kit.edu
%%
%% Version 1.3.3, 2018-04-17

\iflanguage{english}
{\chapter{Appendix}}    % english style
{\chapter{Anhang}}      % german style
\label{chap:appendix}

\section{Results of validation with depth limit}
\label{sec:appendix:depth-limit}

The following tables show the results
when attempting various validations against inferred schemas
with different depth limits.
The meaning of the result column is as follows:

\begin{description}
\item[solved] validation completed successfully (found a solution)
\item[fail] validation completed unsuccessfully (reported violations)
\item[limit] the result is directly “depth limit reached” % TODO perhaps just remove these lines?
\item[core] the process crashed and dumped core (out-of-memory error)
\item[abort] the process was manually killed after a long time with no apparent progress
\end{description}

\begin{table}[ht]
  \centering
  \begin{tabular}{r r r l}
    limit & real time & CPU time & result \\
    \hline
    - & \minsec{2}{50} & \minsec{8}{51} & core \\
    1 & \minsec{0}{0} & \minsec{0}{0} & limit \\
    2 & \minsec{4}{41} & \minsec{17}{22} & core \\
    3 & \minsec{3}{57} & \minsec{13}{31} & core \\
    4 & \minsec{2}{51} & \minsec{9}{1} & core \\
    5 & \minsec{2}{48} & \minsec{8}{54} & core \\
    6 & \minsec{2}{50} & \minsec{8}{58} & core \\
    7 & \minsec{2}{48} & \minsec{8}{41} & core \\
    8 & \minsec{2}{47} & \minsec{8}{44} & core \\
    9 & \minsec{2}{48} & \minsec{8}{48} & core \\
    10 & \minsec{2}{48} & \minsec{8}{53} & core
  \end{tabular}
  \caption{
    Results when validating the item \QL{Q44578}{Titanic}
    against the shape for the class \QL{Q11424}{film}
    from a schema inferred from the set of films that won ten or more Oscars
    (\wdsiJob{29}).
  }
  \label{tab:appendix:depth-limit:1}
\end{table}

\begin{table}[ht]
  \centering
  \begin{tabular}{r r r l}
    limit & real time & CPU time & result \\
    \hline
    - & \minsec{0}{1} & \minsec{0}{3} & fail \\
    1 & \minsec{0}{0} & \minsec{0}{0} & limit \\
    2 & \minsec{0}{16} & \minsec{0}{21} & core \\
    3 & \minsec{0}{1} & \minsec{0}{2} & fail \\
    4 & \minsec{0}{1} & \minsec{0}{3} & fail \\
    5 & \minsec{0}{1} & \minsec{0}{3} & fail \\
    6 & \minsec{0}{1} & \minsec{0}{3} & fail \\
    7 & \minsec{0}{1} & \minsec{0}{3} & fail \\
    8 & \minsec{0}{1} & \minsec{0}{2} & fail \\
    10 & \minsec{0}{1} & \minsec{0}{3} & fail
  \end{tabular}
  \caption{
    Results when validating the item \QL{Q42}{Douglas Adams}
    against the shape for the class \QL{Q5}{human}
    from a schema inferred from the set of films that won ten or more Oscars
    (\wdsiJob{29}).
  }
  \label{tab:appendix:depth-limit:2}
\end{table}

\begin{table}[ht]
  \centering
  \begin{tabular}{r r r l}
    limit & real time & CPU time & result \\
    \hline
    - & \minsec{0}{17} & \minsec{0}{7} & fail \\
    1 & \minsec{0}{0} & \minsec{0}{0} & limit \\
    2 & \minsec{0}{0} & \minsec{0}{1} & solved \\
    3 & \minsec{0}{18} & \minsec{0}{7} & fail \\
    4 & \minsec{0}{19} & \minsec{0}{8} & fail \\
    5 & \minsec{0}{17} & \minsec{0}{7} & fail \\
    6 & \minsec{0}{18} & \minsec{0}{8} & fail \\
    7 & \minsec{0}{18} & \minsec{0}{8} & fail \\
    8 & \minsec{0}{18} & \minsec{0}{8} & fail \\
    9 & \minsec{0}{18} & \minsec{0}{7} & fail \\
    10 & \minsec{0}{18} & \minsec{0}{7} & fail
  \end{tabular}
  \caption{
    Results when validating the item \QL{Q42}{Douglas Adams}
    against the shape for the class \QL{Q5}{human}
    from a schema inferred from the members of the 13th Riigikogu
    (the Estonian parliament; \wdsiJob{30}).
  }
  \label{tab:appendix:depth-limit:3}
\end{table}

\begin{table}[ht]
  \centering
  \begin{tabular}{r r r l}
    limit & real time & CPU time & result \\
    \hline
    - & \minsec{2}{50} & \minsec{8}{51} & core \\
    1 & \minsec{0}{0} & \minsec{0}{0} & limit \\
    2 & \minsec{4}{41} & \minsec{17}{22} & core \\
    3 & \minsec{3}{57} & \minsec{13}{31} & core \\
    4 & \minsec{2}{51} & \minsec{9}{1} & core \\
    5 & \minsec{2}{48} & \minsec{8}{54} & core \\
    6 & \minsec{2}{50} & \minsec{8}{58} & core \\
    7 & \minsec{2}{48} & \minsec{8}{41} & core \\
    8 & \minsec{2}{47} & \minsec{8}{44} & core \\
    9 & \minsec{2}{48} & \minsec{8}{48} & core \\
    10 & \minsec{2}{48} & \minsec{8}{53} & core
  \end{tabular}
  \caption{
    Results when validating the item \QL{Q449851}{Mailis Reps}
    against the shape for the class \QL{Q5}{human}
    from a schema inferred from the members of the 13th Riigikogu
    (\wdsiJob{30}).
  }
  \label{tab:appendix:depth-limit:4}
\end{table}

\begin{table}[ht]
  \centering
  \begin{tabular}{r r r l}
    limit & real time & CPU time & result \\
    \hline
    - & \minsec{1}{25} & \minsec{4}{14} & core \\
    1 & \minsec{0}{1} & \minsec{0}{1} & limit \\
    2 & \minsec{0}{1} & \minsec{0}{1} & fail \\
    3 & \minsec{0}{55} & \minsec{2}{6} & core \\
    4 & \minsec{7}{34} & 1\minsec{3}{54} & core \\
    5 & \minsec{0}{10} & \minsec{0}{27} & core \\
    6 & \minsec{0}{30} & \minsec{1}{5} & core \\
    7 & \minsec{0}{51} & \minsec{1}{59} & core \\
    8 & \minsec{5}{29} & 1\minsec{2}{54} & core \\
    9 & \minsec{0}{9} & \minsec{0}{19} & core \\
    10 & 8\minsec{6}{39} & 15\minsec{3}{35} & abort
  \end{tabular}
  \caption{
    Results when validating the item \QL{Q30}{United States of America}
    against the shape for the class \QL{Q3624078}{sovereign state}
    from a schema inferred from a set of items for bus stops
    (\wdsiJob{15}).
  }
  \label{tab:appendix:depth-limit:5}
\end{table}


\end{document}
