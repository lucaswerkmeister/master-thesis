%% LaTeX2e class for student theses
%% sections/content.tex
%% 
%% Karlsruhe Institute of Technology
%% Institute for Program Structures and Data Organization
%% Chair for Software Design and Quality (SDQ)
%%
%% Dr.-Ing. Erik Burger
%% burger@kit.edu
%%
%% Version 1.3.3, 2018-04-17

% TODO find a better title for this chapter
\chapter{Applying RDF2Graph to Wikidata}
\label{ch:RDF2Graph+Wikidata}

% TODO find a better title for this section
\section{General RDF2Graph Updates}
\label{sec:RDF2Graph+Wikidata:updates}

% TODO find a better title for this section
\section{Wikidata Support}
\label{sec:RDF2Graph+Wikidata:Wikidata}

The most fundamental change to make RDF2Graph support Wikidata
was % TODO is?
to change the type predicates used.
RDF2Graph heavily relies on the type information of the RDF graph it inspects,
assuming a one-to-one mapping between classes and shapes,
and it uses the standard predicates \PName{rdf:type} and \PName{rdfs:subClassOf}
to determine the class of a subject and the superclasses of a class, respectively.
However, Wikidata does not use these standard predicates:
The “instance of” and “subclass of” information of an item
are regular statements like any other Wikidata statement, % TODO the information are? hm
using the properties \P{P31} and \P{P279}, respectively.
In the \gls{rdf} export, \PName{rdf:type} and \PName{rdfs:subClassOf} are only used
as part of the meta-model, % TODO meta-model? hm
assigning each item the class \PName{wikibase:Item}, a subclass of \PName{wikibase:Entity}.
To use the type information within the data instead of this meta-model,
RDF2Graph was changed to use the predicates \PName{wdt:P31} and \PName{wdt:P279} % TODO we haven’t seen the wdt: syntax yet, should that be in the Background?
instead of \PName{rdf:type} and \PName{rdfs:subClassOf}.

Another important aspect is that it’s not feasible to run RDF2Graph on the entirety of Wikidata at once.
For example, the 2018-08-20 full Wikidata dump is \SI{42.7}{\giga\byte} large after gzip compression % TODO >300GB uncompressed (309657344438 bytes), which is probably more impressive
and \num{8316156305} lines long uncompressed,
with most lines corresponding to one triple (except for some blank lines).
More importantly, inferring a single schema from all of Wikidata is also not the intention of this work:
the intention is to infer a schema from a particular set of exemplary items, % TODO awkward passive voice (“*the* intention” is *my* intention, I guess)
and ignore similar items which are not as exemplary (e.~g. not as well maintained)
as well as unrelated items. % TODO that should be explained in more detail in the Introduction, which I haven’t written yet as of this writing
So instead of running RDR2Graph directly against the SPARQL endpoint of the Wikidata Query Service,
a process was set up % TODO super awkward passive voice
to download the data of all selected items as well as the items they directly link to,
load that into a single N-Triples file,
serve that file via a local Fuseki SPARQL server,
run RDF2Graph against that server,
and finally run the ShEx exporter against RDF2Graph’s results.
To make it easier to run, the entire process is controlled by a Makefile in the RDF2Graph repository,
so that after creating a \filename{example.entities.sparql} file with a SPARQL query selecting the exemplary items,
it is sufficient to run \command{make example.shex} to run the entire process and generate the ShEx file.

% TODO find a better title for this section
\section{Schema Reduction}
\label{sec:RDF2Graph+Wikidata:schema-reduction}


% TODO find a better title for this chapter
\chapter{The Wikidata ShEx Inference Tool}
\label{ch:wdsi}

% TODO find a better title for this section
\section{Toolforge Support}
\label{sec:wdsi:Toolforge}

% TODO find a better title for this section
\section{Abstract Considerations}
\label{sec:wdsi:abstract}

% TODO find a better title for this section
\section{Utilities}
\label{sec:wdsi:utilities}
