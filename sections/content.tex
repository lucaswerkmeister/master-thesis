%% LaTeX2e class for student theses
%% sections/content.tex
%% 
%% Karlsruhe Institute of Technology
%% Institute for Program Structures and Data Organization
%% Chair for Software Design and Quality (SDQ)
%%
%% Dr.-Ing. Erik Burger
%% burger@kit.edu
%%
%% Version 1.3.3, 2018-04-17

% TODO find a better title for this chapter
\chapter{Applying RDF2Graph to Wikidata}
\label{ch:RDF2Graph+Wikidata}

% TODO find a better title for this section
\section{General RDF2Graph Updates}
\label{sec:RDF2Graph+Wikidata:updates}

% TODO find a better title for this section
\section{Wikidata Support}
\label{sec:RDF2Graph+Wikidata:Wikidata}

The most fundamental change to make RDF2Graph support Wikidata
was % TODO is?
to change the type predicates used.
RDF2Graph heavily relies on the type information of the RDF graph it inspects,
assuming a one-to-one mapping between classes and shapes,
and it uses the standard predicates \PName{rdf:type} and \PName{rdfs:subClassOf}
to determine the class of a subject and the superclasses of a class, respectively.
However, Wikidata does not use these standard predicates:
The “instance of” and “subclass of” information of an item
are regular statements like any other Wikidata statement, % TODO the information are? hm
using the properties \P{P31} and \P{P279}, respectively.
In the \gls{rdf} export, \PName{rdf:type} and \PName{rdfs:subClassOf} are only used
as part of the meta-model, % TODO meta-model? hm
assigning each item the class \PName{wikibase:Item}, a subclass of \PName{wikibase:Entity}.
To use the type information within the data instead of this meta-model,
RDF2Graph was changed to use the predicates \PName{wdt:P31} and \PName{wdt:P279} % TODO we haven’t seen the wdt: syntax yet, should that be in the Background?
instead of \PName{rdf:type} and \PName{rdfs:subClassOf}.

Another important aspect is that it’s not feasible to run RDF2Graph on the entirety of Wikidata at once.
For example, the 2018-08-20 full Wikidata dump is \SI{42.7}{\giga\byte} large after gzip compression % TODO >300GB uncompressed (309657344438 bytes), which is probably more impressive
and \num{8316156305} lines long uncompressed,
with most lines corresponding to one triple (except for some blank lines).
More importantly, inferring a single schema from all of Wikidata is also not the intention of this work:
the intention is to infer a schema from a particular set of exemplary items, % TODO awkward passive voice (“*the* intention” is *my* intention, I guess)
and ignore similar items which are not as exemplary (e.~g. not as well maintained)
as well as unrelated items. % TODO that should be explained in more detail in the Introduction, which I haven’t written yet as of this writing
So instead of running RDR2Graph directly against the SPARQL endpoint of the Wikidata Query Service,
a process was set up % TODO super awkward passive voice
to download the data of all selected items as well as the items they directly link to,
load that into a single N-Triples file,
serve that file via a local Fuseki SPARQL server,
run RDF2Graph against that server,
and finally run the ShEx exporter against RDF2Graph’s results.
To make it easier to run, the entire process is controlled by a Makefile in the RDF2Graph repository,
so that after creating a \filename{example.entities.sparql} file with a SPARQL query selecting the exemplary items,
it is sufficient to run \command{make example.shex} to run the entire process and generate the ShEx file.

This data download step also presents a convenient opportunity to drop some data from the items.
For one, the labels, descriptions and aliases of an item can make up a large portion of its data,
but are generally not interesting for ShEx schemas,
since there is rarely more to say about them than
“an item has one or more labels, zero or more descriptions, and zero or more aliases”. % TODO express as ShExC?
% TODO in theory it’s possible for items to have zero labels…
(A manually curated schema might go beyond this –
perhaps requiring, for example, that items about Spanish municipalities have a label in Spanish –
but such details are beyond the ability of RDF2Graph to infer.)
Additionally, Wikidata items often have a number of statements listing external identifiers for the item
(that is, identifiers in other databases for the same concept described by this item):
for example, Wikidata’s \Q{Q80} corresponds to \loc{no99010609} in the \gls{loc},
\viaf{85312226} in the \gls{viaf},
\imdbName{nm3805083} in the \gls{imdb},
\TwitterAccount{timberners\_lee} on Twitter,
and dozens of other external identifiers. % TODO is a vague “dozens” okay?
All these external identifiers are technically just ordinary statements,
but since they are generally not very interesting compared to other statements,
they are sorted into a separate section when viewing the item on the Wikidata website.
To reduce the runtime of the RDF2Graph inference process,
and to reduce clutter in the inferred schemas,
the labels, descriptions, aliases, and external identifiers of an item
are therefore excluded when downloading the data for a set of exemplary items.

However, one unfortunate consequence of running RDF2Graph against a reduced data set
is that RDF2Graph cannot see the full type hierarchy of the classes involved:
for example, depending on how much data was downloaded,
it may or may not be aware that the classes \QL{Q112099}{island nation} and \QL{Q3624078}{sovereign state}
have a common superclass, \QL{Q7275}{state},
which reduces the available options during the simplification step. % TODO not happy with this phrasing
To mitigate this, RDF2Graph was patched % TODO awkward passive voice
to allow specifying an alternative SPARQL endpoint for all queries that require a full view of the data,
and uses this alternative endpoint for the query to get all parent and child classes of a class.
The Makefile-controlled process mentioned above % TODO awkward phrasing
specifies the Wikidata Query Service SPARQL endpoint for this option,
so that RDF2Graph can learn the full class hierarchy around all relevant items % TODO is “learn” the right word?
even when running against a subset of Wikidata. % TODO “subset of Wikidata”? not quite happy with that phrasing

The simplification process also required some changes to support Wikidata.
Originally, RDF2Graph would merge all classes into their superclass,
almost completely unconditionally,
stopping only at the root class \PName{owl:Thing}.
This was much too aggressive for Wikidata:
not only does Wikidata have a different root class (\QL{Q35120}{entity}),
but it also has a complex hierarchy of abstract classes below that root class,
and merging other classes into those abstract classes
(like \QL{Q151885}{concept}, \QL{Q7184903}{abstract object} or \QL{Q830077}{subject})
would not result in useful schemas.
Therefore, the merging step in the simplification process was adapted % TODO awkward passive voice
to add a number of Wikidata items to the set of classes which other classes should not be merged into
(originally only containing \PName{owl:Thing}),
and to stop looking for common superclasses of two classes after checking five levels in the class hierarchy.
(The limit of five levels is an arbitrary choice
and could also be made configurable if deemed necessary in the future.) % TODO look a bit more into this?
% TODO also disabled step 7.4.4 (dropping predicates that were found in superclasses)
% because the superclass support in the ShEx export changed,
% but that requires explanation of how the ShEx export changed

% TODO find a better title for this section
\section{Schema Reduction}
\label{sec:RDF2Graph+Wikidata:schema-reduction}


% TODO find a better title for this chapter
\chapter{The Wikidata ShEx Inference Tool}
\label{ch:wdsi}

% TODO find a better title for this section
\section{Toolforge Support}
\label{sec:wdsi:Toolforge}

% TODO find a better title for this section
\section{Abstract Considerations}
\label{sec:wdsi:abstract}

% TODO find a better title for this section
\section{Utilities}
\label{sec:wdsi:utilities}
