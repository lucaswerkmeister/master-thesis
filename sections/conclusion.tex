%% LaTeX2e class for student theses
%% sections/conclusion.tex
%% 
%% Karlsruhe Institute of Technology
%% Institute for Program Structures and Data Organization
%% Chair for Software Design and Quality (SDQ)
%%
%% Dr.-Ing. Erik Burger
%% burger@kit.edu
%%
%% Version 1.3.3, 2018-04-17

\chapter{Conclusion}
\label{ch:Conclusion}

The goal of this thesis was to investigate how \gls{shex} \glspl{schema} can be automatically inferred for \gls{Wikidata},
and how useful the resulting \glspl{schema} are.
This was done using an updated and adapted version of the \gls{RDF2Graph} software,
which was made available to the \gls{Wikidata} community
through a new web-based tool.

In addition to the changes specific to \gls{Wikidata},
many general improvements to \gls{RDF2Graph} were made over the course of this thesis,
making it easier to use and more robust on any graph.
All these changes are available under the same free software license as the original \gls{RDF2Graph},
and I hope that some of them will be included in the original source code repository in the future.

When attempting to validate other \glspl{item} against the inferred \glspl{schema},
an unexpected problem arose:
none of the existing \gls{shex} validators were able to reliably perform the validation without crashing.
Several strategies were attempted to remediate this,
both in the \gls{schema} extraction and in the validators,
but this was ultimately unsuccessful.

However, this does not mean the \glspl{schema} are not useful.
Sometimes, problems in the input data can manifest themselves
in the form of unusual \glspl{predicate} or target classes in a \gls{schema},
which an attentive reader can notice and trace back to the problematic \glspl{item} in the input.
And the full, automatically inferred \glspl{schema}
can also form the basis for shorter \glspl{schema} manually extracted from the longer ones,
which can either be validated directly
(now without problems from the validators)
or be further refined by users familiar with the data model,
making the automatically inferred \glspl{schema} a useful basis for manually curated \glspl{schema}.

There are plenty of options of further improvements on this work.
\Cref{sec:Evaluation:duration} already mentioned how the \gls{wdsi} could be improved
to notify the user of the estimated total runtime of a \gls{job} once the data download step is complete
and to reject \glspl{job} which are expected to take too long.
Many other improvements could be made to the user interface as well,
such as exposing some more configuration options to users of the tool,
e.~g. the thresholds for \gls{schema} reduction mentioned in \cref{sec:RDF2Graph+Wikidata:schema-reduction}.

A significant improvement over the current state
would be to make \gls{RDF2Graph} work on full \gls{statement} nodes instead of just the “truthy” \glspl{statement}.
A full explanation of full \gls{statement} nodes is beyond the scope of this thesis,
but in brief, they offer much more information about \gls{Wikidata} \glspl{statement}
in exchange for a slightly more complicated data format.
Using full \gls{statement} nodes would allow \gls{RDF2Graph} to take not just the \gls{statement}’s values
but also their \glspl{qualifier} and \glspl{reference} into account.
However, this would require major changes to the way \gls{RDF2Graph} looks at the input graph,
because the \gls{subject} and \gls{object} of a \gls{statement} are no longer linked via a single \gls{triple}
in the full \gls{statement} nodes.

Clearly, it is not a satisfactory final state
that the inferred \glspl{schema} cannot be directly used for validation.
It may be possible to find optimizations in the validators
and/or useful criteria for reducing the inferred \glspl{schema}
which would enable direct validation against the \glspl{schema}
without human intervention to manually extract their most useful parts.
There is also some room for improvement in the simplification step
regardless of the ability to validate against the inferred \glspl{schema}:
sometimes, despite the changes in \cref{subsec:RDF2Graph+Wikidata:Wikidata:simplification},
it still merges mostly-unrelated classes into very abstract base classes,
so the criteria on when to merge or not merge classes could use some improvements.
Additionally, it may be useful to discard rarely used classes for common \glspl{type link} altogether,
to cut down on the amount of “noise” in the schema.

The detailed timings for the jobs listed in \cref{sec:jobs-over-various}
show that the \gls{shex} export step usually takes up about half of the total wall-clock time,
and often significantly more than half of the CPU time.
I have not investigated this step more closely except for its very last part
(see \cref{sec:RDF2Graph+Wikidata:updates}),
but it seems unlikely that this is necessary:
it may be possible to change or rewrite the rest of the \gls{shex} exporter to be much more efficient.

% TODO finish off conclusion?
