%% LaTeX2e class for student theses
%% sections/content.tex
%% 
%% Karlsruhe Institute of Technology
%% Institute for Program Structures and Data Organization
%% Chair for Software Design and Quality (SDQ)
%%
%% Dr.-Ing. Erik Burger
%% burger@kit.edu
%%
%% Version 1.3.3, 2018-04-17

\chapter{Background}
\label{ch:Background}


\section{Wikidata}
\label{ch:Background:Wikidata}

Explain the Wikidata data model,
the central role of the community,
and perhaps some fundamental properties.

Also explain the RDF export
and the query service,
and how they’re derived from Wikidata,
not the canonical source of truth.


\section{Shape Expressions}
\label{ch:Background:ShEx}

\acrfull{shex} \cite{shex}
is a standard for describing data shapes within an RDF graph,
developed by the ShEx Community Group under the umbrella of the \gls{w3c}.
A ShEx schema consists of a number of shapes,
each of which describes a focus node,
expressing restrictions on its node kind, value, and/or incoming and outgoing links.

For example, the schema in \cref{listing:shex-example} defines two shapes,
one for humans and one for locations,
in a fictional example vocabulary.
The shape for humans describes that a human has one or more parents,
any number of children,
a date of birth,
and a place of birth;
the parents and children should also be humans,
the date of birth should be a date literal,
and the place of birth should be a location.
The shape for locations describes that a location may optionally be contained within another location.

\begin{lstfloat}
\begin{lstlisting}[language=sparql]
PREFIX ex: <http://shex.example/>

ex:Human {
  ex:parent @ex:Human+;
  ex:child @ex:Human*;
  ex:dateOfBirth xsd:dateTime;
  ex:placeOfBirth @ex:Location;
}

ex:Location {
  ex:containedWithin @ex:Location?;
}
\end{lstlisting}
\caption{A simple example schema.}
\label{listing:shex-example}
% TODO this schema is impossible to satisfy without a circular parent relationship – not a great example :/
\end{lstfloat}

How they work.

Difference between v1 and v2?


\section{RDF2Graph}
\label{ch:Background:RDF2Graph}

Summarize how RDF2Graph works?
The distinction between the steps
(main, simplify, ShEx export)
is likely to be significant for the rest of the thesis.
