%% LaTeX2e class for student theses
%% sections/content.tex
%% 
%% Karlsruhe Institute of Technology
%% Institute for Program Structures and Data Organization
%% Chair for Software Design and Quality (SDQ)
%%
%% Dr.-Ing. Erik Burger
%% burger@kit.edu
%%
%% Version 1.3.3, 2018-04-17

\chapter{Background}
\label{ch:Background}


\section{Wikidata}
\label{ch:Background:Wikidata}

Wikidata is a free knowledge base
and part of the Wikimedia family of sister projects,
the most famous of which is Wikipedia, the free encyclopedia.
Its contents are created, maintained and managed by the Wikidata community,
most of whose members are volunteers,
as well as the members of Wikidata’s sister projects, e.~g. Wikipedia.
Anyone can contribute to Wikidata,
but the community ensures the quality of the contents with various quality control mechanisms.
Providing another such mechanism is part of the motivation for this thesis. % TODO for this work?

Information on Wikidata is collected in items,
which represent things or concepts:
there are items for individual persons,
for cities, states, geographical features,
for organizations for corporations,
items for books, films, newspapers, journals, scientific articles,
for abstract concepts, phenomena, emotions, philosophical movements, political orientations,
items for conceptual hierarchies, parent classes, biological taxa,
and even items for fictional characters, places, or other entities.

All of these items follow the same structure:
they are identified by their item ID,
a consecutive number prefixed with the letter “Q”
(e.~g. \Q{Q25291} for the year 2018),
have a label, description, and search aliases in various languages
(for example, \Q{Q7251} is “Alan Turing” in English but «\foreignlanguage{russian}{Алан Тьюринг}» in Russian),
a set of sitelinks (links to pages about the same concept in various other Wikimedia projects),
and most importantly, a set of statements.

The statements are where most of the information in Wikidata is stored.
They consist of a property, such as “place of birth” or “author” or “population”,
and a value, which can be a reference to another item, a quantity, a point in time, a piece of text,
or a few other possible types.
A statement can also have qualifiers
(further property-value pairs, e.~g. clarifying when or where the statement is valid)
and references (sets of property-value pairs, listing sources for the statement),
but those are mostly ignored in the context of this thesis.
Properties are also identified by an ID
(prefixed with the letter “P” instead of “Q”)
and also have labels, descriptions and aliases in different languages:
for example, \P{P31} is “instance of” in English and „\foreignlanguage{ngerman}{ist ein(e)}“ in German.
The labels and descriptions are necessary to understand the meaning of statements,
but they are not themselves part of the statements:
statements only list references to property and item IDs,
making most of the information in Wikidata language-agnostic.

It is vital to observe that there is no kind of schema inherent to the Wikidata data model.
Any property can be used on any item:
nothing in the software stops one from adding, say,
a “date of birth” statement to an item for a lake,
or a “parent taxon” statement to an item for a movie teaser poster.
The community has several ways to describe schemas to varying degrees of formality
(such as property lists on WikiProject pages or the property constraints system),
but they are all realized by community consensus,
not enforced by the Wikidata software.
The great flexibility which this lends to the community has proven to be one of Wikidata’s greatest strengths, % TODO citation?
and though the use of shape expressions % TODO acronym?
on Wikidata will provide another,
highly formal way to describe schemas,
it is not intended to change this fundamental operating principle of Wikidata. % TODO awkward passive voice

The Wikibase data model, as described above,
is not directly related to RDF. % TODO acronym
However, to enable usage of RDF technologies and interoperation with RDF-based datasets,
Wikidata’s data is exported to RDF:
the data about any item can be downloaded in various RDF formats through a linked data interface, % TODO *the* linked data interface?
and a full, up-to-date RDF export of Wikidata is available in the Wikidata Query Service,
a SPARQL endpoint to which anyone may submit queries.
However, this interface is export-only:
it is not possible to edit Wikidata via RDF.



Explain the Wikidata data model,
the central role of the community,
and perhaps some fundamental properties.

Also explain the RDF export
and the query service,
and how they’re derived from Wikidata,
not the canonical source of truth.


\section{Shape Expressions}
\label{ch:Background:ShEx}

\acrfull{shex} \cite{shex}
is a standard for describing data shapes within an RDF graph,
developed by the ShEx Community Group under the umbrella of the \gls{w3c}.
A ShEx schema consists of a number of shapes,
each of which describes a focus node,
expressing restrictions on its node kind, value, and/or incoming and outgoing links.

For example, the schema in \cref{listing:shex-example} defines two shapes,
one for humans and one for locations,
in a fictional example vocabulary.
The shape for humans describes that a human has one or more parents,
any number of children,
a date of birth,
and a place of birth;
the parents and children should also be humans,
the date of birth should be a date literal,
and the place of birth should be a location.
The shape for locations describes that a location may optionally be contained within another location.

\begin{lstfloat}
\begin{lstlisting}[language=sparql]
PREFIX ex: <http://shex.example/>

ex:Human {
  ex:parent @ex:Human+;
  ex:child @ex:Human*;
  ex:dateOfBirth xsd:dateTime;
  ex:placeOfBirth @ex:Location;
}

ex:Location {
  ex:containedWithin @ex:Location?;
}
\end{lstlisting}
\caption{A simple example schema.}
\label{listing:shex-example}
% TODO this schema is impossible to satisfy without a circular parent relationship – not a great example :/
\end{lstfloat}

How they work.

Difference between v1 and v2?


\section{RDF2Graph}
\label{ch:Background:RDF2Graph}

Summarize how RDF2Graph works?
The distinction between the steps
(main, simplify, ShEx export)
is likely to be significant for the rest of the thesis.
