%% LaTeX2e class for student theses
%% sections/evaluation.tex
%% 
%% Karlsruhe Institute of Technology
%% Institute for Program Structures and Data Organization
%% Chair for Software Design and Quality (SDQ)
%%
%% Dr.-Ing. Erik Burger
%% burger@kit.edu
%%
%% Version 1.3.3, 2018-04-17

\chapter{Evaluation}
\label{ch:Evaluation}

% TODO find a better title for this section
\section{Schema Quality}
\label{sec:Evaluation:quality}

In general, the quality of the inferred schemas is satisfactory.
On occasion, parts of the schema may not make sense –
for example, if a shape declares that the value for a \PL{P735}{given name} statement
should be a \QL{Q202444}{given name} or a \QL{Q101352}{family name} –
but these problems are usually not the fault of the inference process,
but rather point to problems in the input data:
in this example, someone probably used an item for a family name
instead of the identical given name. % TODO example!
% TODO more about the schema quality (with examples!)
% TODO elaborate

% TODO find a better title for this section
\section{Duration of the inference process}
\label{sec:Evaluation:duration}

While attempts to validate data against the inferred schema
suffer due to the size of the input data and the schemas,
no such problems appear to plague the inference process,
which, while slow, has a more reliable runtime.
The various jobs that were run on the Wikidata Shape Expressions Inference Tool,
whose execution times range from five or ten minutes to several hours,
generally show a linear relationship between the number of triples in the input data
and the total duration of the inference job
(see \cref{fig:jobs-over-triples}).
It is expected that this linear relationship will continue to hold for larger input data sets,
and that there is no fundamental limit barring the inference of ever larger data sets,
provided one is willing to wait long enough for the results.
% TODO add the data – here or in an appendix?

It should be noted % TODO awkward passive voice
that this is a relation between the number of triples and the execution time,
but the triples are not directly the input of the inference process:
that input is a SPARQL query which selects a set of input \emph{entities},
and the triples are the statements of those entities and the entities they link to,
downloaded in an early step of the inference process.
The relationship between the \emph{number of entities} selected and the execution time
is much more chaotic
(see \cref{fig:jobs-over-entities}), % TODO add the data for this as well?
since it highly depends on the selected entities:
how many statements they themselves have,
as well as how many other entities they link to.
However, the data download step is itself a fairly cheap % TODO “cheap” acceptable?
step of the inference process,
so it is not very difficult to derive the number of triples,
from which it is then possible to predict the total execution type of the inference process fairly reliably.

% TODO describe where the data (#entities, #lines, job execution times) comes from?
% to what level of detail? should I mention the custom script to read the accounting file backwards?

% TODO with the latest jobs, the data doesn’t really fit the text anymore :(

\begin{figure}
  \centering
  \input{graphs/jobs-over-triples}
  \caption[A chart of job execution time over number of triples in the input data set.]{
    A chart of job execution time over number of triples in the input data set,
    along with three different functions fit to the data.
    All three fits resulted in roughly the same linear function
    and approximate the data reasonably well.
    % TODO explain the “0.00” x² coefficient?
  }
  \label{fig:jobs-over-triples}
\end{figure}

\begin{figure}
  \centering
  \input{graphs/jobs-over-entities}
  \caption[A chart of job execution time over number of entities in the input data set.]{
    A chart of job execution time over number of entities in the input data set,
    along with three different functions fit to the data.
    The three fits resulted in very different functions
    and don’t approximate the data very well;
    in fact, the quadratic function appears to suggest
    that execution time shall become negative if one only supplies enough entities,
    which is clearly nonsensical. % TODO is “clearly nonsensical” appropriate style?
    % TODO improve wording so it’s not quite as derivative of the previous figure
  }
  \label{fig:jobs-over-entities}
\end{figure}

This suggests two possible future improvements
for the Wikidata Shape Expressions Inference Tool:
to report to the user how long a job is expected to take,
as soon as the download step has finished;
and to reject jobs which resulted in too much data,
and are expected to take far too long to be tolerable.
Currently, the tool merely suggests to its users
that their queries should not select more than about fifty entities,
but does not implement any kind of hard limit beyond that.

% TODO user feedback for the tool

% TODO unable to validate directly

% TODO useful as a basis for manually assembled schemas (which can then be used to validate)
