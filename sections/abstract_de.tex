%% LaTeX2e class for student theses
%% sections/abstract_de.tex
%% 
%% Karlsruhe Institute of Technology
%% Institute for Program Structures and Data Organization
%% Chair for Software Design and Quality (SDQ)
%%
%% Dr.-Ing. Erik Burger
%% burger@kit.edu
%%
%% Version 1.3.3, 2018-04-17

\Abstract
\Gls{Wikidata}, die freie Wissensdatenbank in der \gls{Wikimedia}-Bewegung,
wird von verschiedenen \gls{Wikimedia}- und anderen Projekten
als Quelle für maschinenlesbare Informationen und Daten verwendet.
Die Datenqualität wird durch die \gls{Wikidata}-Community verwaltet und überwacht,
wobei verschiedene Mechanismen zur Qualitätskontrolle zum Einsatz kommen,
in letzter Zeit auch formale Schemata in der \acrlong{shex}-Sprache.
Allerdings ist es langwierig, größere Schemata zu schreiben,
was automatischen Rückschluss solcher Schemata aus einem Satz beispielhafter Datenobjekte
attraktiv macht.

Diese Arbeit untersucht diese Option,
indem das \gls{RDF2Graph}-Programm aktualisiert und angepasst wird,
um Schemata aus einem Satz von \gls{Wikidata}-Datenobjekten rückzuschließen,
und durch das Angebot eines webbasierten Werkzeugs,
welches diesen Vorgang der \gls{Wikidata}-Community zugänglich macht.
Obwohl die resultierenden Schemata meist nicht für direkte Validierung geeignet sind,
können sie immer noch als Beschreibung eines Bereichs des \gls{Wikidata}-Datenmodells,
als Mittel, mögliche Probleme in den Eingabedaten zu bemerken,
oder als Basis für manuell betreute Schemata nützlich sein.
