% TODO define terms for Label, Description, Statement, etc.,
% then write them all in uppercase per Lydia’s comment
\newacronym{imdb}{IMDb}{Internet Movie Database}
\newacronym{loc}{LoC}{Library of Congress}
\newacronym{rdf}{RDF}{Resource Description Framework}
\newacronym[longplural={Requests for Comments}]{rfc}{RfC}{Request for Comments}
\newacronym{shex}{ShEx}{Shape Expressions}
\newacronym{shexc}{ShExC}{\acrshort{shex} Compact Syntax}
\newacronym{sparql}{SPARQL}{\acrshort{sparql} Protocol and \acrshort{rdf} Query Language}
\newacronym[first={the Virtual International Authority File (VIAF)},long={the Virtual International Authority File}]{viaf}{VIAF}{Virtual International Authority File}
\newacronym{w3c}{W3C}{World Wide Web Consortium}
\newacronym[first={the Wikidata Query Service (WDQS)},long={the Wikidata Query Service}]{wdqs}{WDQS}{Wikidata Query Service}
\newacronym[description={Internationalized Resource Identifier, Unicode-aware generalization of \acrshortpl{uri}}]{iri}{IRI}{Internationalized Resource Identifier}
\newacronym[description={Uniform Resource Identifier, usually a \acrshort{url}}]{uri}{URI}{Uniform Resource Identifier}
\newacronym{url}{URL}{Uniform Resource Locator}
\newacronym{html}{HTML}{Hypertext Markup Language}
\newacronym{css}{CSS}{Cascading Style Sheets}
\newacronym{http}{HTTP}{Hypertext Transfer Protocol}
\newacronym{xsd}{XSD}{XML Schema Definition}

% never use the long form of these acronyms unless explicitly requested
\glsunset{sparql}
\glsunset{url}
\glsunset{html}
\glsunset{css}
\glsunset{http}

\newglossaryentry{Wikimedia}{
  name=Wikimedia,
  description={free knowledge movement and community},
}
\newglossaryentry{Wikidata}{
  name=Wikidata,
  description={free knowledge base in the \gls{Wikimedia} movement},
}
\newglossaryentry{Wikipedia}{
  name=Wikipedia,
  description={free encyclopedia in the \gls{Wikimedia} movement},
}
\newglossaryentry{Wikimedia Commons}{
  name={Wikimedia Commons},
  description={free media repository in the \gls{Wikimedia} movement},
}
\newglossaryentry{Wikiquote}{
  name=Wikiquote,
  description={free quotation collection in the \gls{Wikimedia} movement},
}
\newglossaryentry{item}{
  name=item,
  description={representation of a thing or concept in \gls{Wikidata}},
}
\newglossaryentry{item ID}{
  name={item ID},
  description={unique, language-agnostic identifier of an \gls{item}, a consecutive number prefixed with the letter “Q”},
}
\newglossaryentry{property}{
  name=property,
  plural=properties,
  description={a possible feature, characteristic or quality of an \gls{item}},
}
\newglossaryentry{property ID}{
  name={property ID},
  description={unique, language-agnostic identifier of a \gls{property}, a consecutive number prefixed with the letter “P”},
}
\newglossaryentry{label}{
  name=label,
  description={the primary term for an \gls{item} or \gls{property} in a language},
}
\newglossaryentry{description}{
  name=description,
  description={a short clarifying or disambiguating text for an \gls{item} or \gls{property}},
}
\newglossaryentry{alias}{
  name=alias,
  plural=aliases,
  description={additional terms by which an \gls{item} or \gls{property} should be found},
}
\newglossaryentry{sitelink}{
  name=sitelink,
  description={a link from an \gls{item} to a page in a \gls{Wikimedia} project about the thing or concept the \gls{item} represents},
}
\newglossaryentry{statement}{
  name=statement,
  description={a unit of information in \gls{Wikidata}, consisting of a subject \gls{item}, a predicate \gls{property}, and a value (\gls{item ID}, quantity, point in time, etc.)},
}
\newglossaryentry{qualifier}{
  name=qualifier,
  description={an additional \gls{property}-value pair further refining a \gls{statement}},
}
\newglossaryentry{reference}{
  name=reference,
  description={a collection of \gls{property}-value pairs recording a source for a \gls{statement}},
}
\newglossaryentry{wdsi}{
  name={Wikidata Shape Expressions Inference tool},
  description={web-based tool to make the inference process available to the \gls{Wikidata} community},
}
\newglossaryentry{triple}{
  name=triple,
  description={a unit of information in \gls{rdf}, consisting of a \gls{subject} \gls{resource}, a \gls{predicate} \gls{resource}, and an \gls{object} value (\gls{resource} or literal)},
}
\newglossaryentry{subject}{
  name=subject,
  description={the first element of an \gls{rdf} \gls{triple}, the “<Alan Turing>” in “<Alan Turing> <is a> <human>”}
}
\newglossaryentry{predicate}{
  name=predicate,
  description={the second element of an \gls{rdf} \gls{triple}, the “<is a>” in “<Alan Turing> <is a> <human>”}
}
\newglossaryentry{object}{
  name=object,
  description={the third element of an \gls{rdf} \gls{triple}, the “<human>” in “<Alan Turing> <is a> <human>”}
}
\newglossaryentry{resource}{
  name=resource,
  description={representation of a thing or concept in \gls{rdf}, identified by an \gls{iri}},
}
\newglossaryentry{schema}{
  name=schema,
  description={formal description of the structure of a linked data set; in \gls{shex}, a collection of \glspl{shape}},
}
\newglossaryentry{shape}{
  name=shape,
  description={element of a \gls{shex} \gls{schema}, describing the structure of one node category},
}
\newglossaryentry{focus node}{
  name={focus node},
  description={the \gls{rdf} \gls{resource} currently being matched against a \gls{shex} \gls{shape}},
}
\newglossaryentry{triple constraint}{
  name={triple constraint},
  description={element of a \gls{shex} \gls{shape}, restricting \glspl{triple} with the \gls{focus node} as the \gls{subject} and a certain \gls{predicate}},
}
\newglossaryentry{value constraint}{
  name={value constraint},
  description={element of a \gls{shex} \gls{triple constraint}, restricting the \gls{object} (value) of a \gls{triple} to match a certain datatype or other \gls{shape}},
}
\newglossaryentry{RDF2Graph}{
  name={RDF\oldstylenums{2}Graph},
  description={tool to automatically determine a \gls{schema} from an \gls{rdf} graph},
}
\newglossaryentry{RDFSimpleCon}{
  name=RDFSimpleCon,
  description={library used by \gls{RDF2Graph}},
}
