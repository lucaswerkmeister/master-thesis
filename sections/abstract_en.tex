%% LaTeX2e class for student theses
%% sections/abstract_en.tex
%% 
%% Karlsruhe Institute of Technology
%% Institute for Program Structures and Data Organization
%% Chair for Software Design and Quality (SDQ)
%%
%% Dr.-Ing. Erik Burger
%% burger@kit.edu
%%
%% Version 1.3.3, 2018-04-17

\Abstract
\Gls{Wikidata}, the free knowledge base in the \gls{Wikimedia} movement,
is used by various \gls{Wikimedia} projects and third parties to provide machine-readable information and data.
Its data quality is managed and monitored by its community using several quality control mechanisms,
recently including formal \glspl{schema} in the \acrlong{shex} language.
However, larger \glspl{schema} can be tedious to write,
making automatic inference of \glspl{schema} from a set of exemplary \glspl{item}
an attractive prospect.

This thesis investigates this option
by updating and adapting the \gls{RDF2Graph} program
to infer \glspl{schema} from a set of \gls{Wikidata} \glspl{item},
and providing a web-based tool which makes this process available to the \gls{Wikidata} community.
Though the resulting \glspl{schema} are usually not fit for direct validation,
they can still be useful as a form of describing the layout of an area of \gls{Wikidata}’s data model,
a way to notice potential issues in the source data,
or a basis for a manually curated \gls{schema}.
