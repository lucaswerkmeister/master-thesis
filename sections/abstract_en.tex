%% LaTeX2e class for student theses
%% sections/abstract_en.tex
%% 
%% Karlsruhe Institute of Technology
%% Institute for Program Structures and Data Organization
%% Chair for Software Design and Quality (SDQ)
%%
%% Dr.-Ing. Erik Burger
%% burger@kit.edu
%%
%% Version 1.3.3, 2018-04-17

\Abstract
% TODO should be longer (1-2 paragraphs, not more than half a page), including a few sentences about Wikidata and its importance, the problem, and my contribution
\acrlong{shex} are a promising additional quality control mechanism for \gls{Wikidata},
the free knowledge base in the \gls{Wikimedia} movement,
but larger \glspl{schema} can be tedious to write,
making automatic inference of \glspl{schema} from a set of exemplary \glspl{item} an attractive prospect. % TODO is “prospect” a good word here?
This thesis investigates % TODO investigated? which tense?
this option by updating and adapting the \gls{RDF2Graph} program to infer \glspl{schema} from a set of \gls{Wikidata} \glspl{item},
and provides a web-based tool which makes this process available to the \gls{Wikidata} community.
Though the resulting \glspl{schema} are usually not fit for direct validation,
they can still be useful as a form of describing the layout of an area of \gls{Wikidata}’s data model,
a way to notice potential issues in the source data,
or a basis for a manually curated \gls{schema}.
