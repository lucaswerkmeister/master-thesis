%% LaTeX2e class for student theses
%% sections/abstract_en.tex
%% 
%% Karlsruhe Institute of Technology
%% Institute for Program Structures and Data Organization
%% Chair for Software Design and Quality (SDQ)
%%
%% Dr.-Ing. Erik Burger
%% burger@kit.edu
%%
%% Version 1.3.3, 2018-04-17

\Abstract
% TODO should be longer (1-2 paragraphs, not more than half a page), including a few sentences about Wikidata and its importance, the problem, and my contribution
To provide an additional quality control mechanism for Wikidata,
a tool to automatically infer data schemas from a set of exemplary Wikidata items
was developed and evaluated.
Though the resulting schemas are usually not fit for direct validation,
they can still be useful as a form of describing the layout of an area of Wikidata’s data model,
a way to see potential issues in the source data, % TODO “see” sounds weak, but “highlight” is too strong a word IMHO given that you have to comb through the schema to find issues
or a basis for a manually curated schema.
