%% LaTeX2e class for student theses
%% sections/content.tex
%% 
%% Karlsruhe Institute of Technology
%% Institute for Program Structures and Data Organization
%% Chair for Software Design and Quality (SDQ)
%%
%% Dr.-Ing. Erik Burger
%% burger@kit.edu
%%
%% Version 1.3.3, 2018-04-17

\chapter{Introduction}
\label{ch:Introduction}

As \gls{Wikidata},
the free knowledge base in the \gls{Wikimedia} movement,
continues to grow in volume and scope \cite{wikidata-fifth-birthday}
and is used by more and more \gls{Wikimedia} projects and third parties,
its data quality has been identified as one of the most important areas of development in the future \cite{wdcon2017-sotp}:
in order for \gls{Wikidata} to be useful,
its data must be trustworthy and available in a consistent format.
Unchecked vandalism discourages data reuse,
while inconsistent data models make it significantly more difficult or even impossible.

To combat these problems,
several quality control mechanisms are used on \gls{Wikidata}.
Recently, editors have begun exploring the use of \acrlong{shex}
as another quality control mechanism to use,
forming the \href{https://www.wikidata.org/wiki/Wikidata:WikiProject_ShEx}{WikiProject \acrshort{shex}}.
Compared to the more established, \gls{Wikidata}-specific quality constraints system,
\acrlong{shex} are more powerful and expressive,
and are also not specific to \gls{Wikidata} alone.
However, \glspl{schema} for \acrlong{shex} are tedious to write by hand.

Automatically inferring \glspl{schema} from \gls{Wikidata} \glspl{item}
promises to simplify the \gls{schema} authoring process:
instead of manually putting together the \gls{schema},
describing shapes for different classes of \glspl{item},
one simply selects a set of \glspl{item},
and a \gls{schema} is automatically generated based on the data about these \glspl{item}.
If the selected \glspl{item} have been carefully edited
to conform to a pre-existing \gls{schema},
perhaps described informally or only present in the minds of the editors,
then the result may be a formalization of that \gls{schema};
alternatively, applying the same process to a less curated set of input \glspl{item}
may result in a coherent summary of the current \gls{schema} of those \glspl{item}
and possibly even demonstrate problems in the input data.

This thesis investigates the usefulness and applicability
of automatically inferring \glspl{schema} for \gls{Wikidata} from sets of exemplary \glspl{item}.
It builds on the existing \gls{RDF2Graph} \cite{vanDam2015} program,
updating and adapting it to support \gls{Wikidata}
and automating the whole inference process.
This is then made available to the whole \gls{Wikidata} community
by incorporating it into a web-based tool.

The remainder of this thesis
is organized as follows.
\Cref{ch:Background} explains concepts that are required
to understand the rest of the thesis.
\Cref{ch:RDF2Graph+Wikidata} describes general updates for \gls{RDF2Graph}
as well as changes that were made to add \gls{Wikidata} support to it.
\Cref{ch:wdsi} introduces the \gls{wdsi}
and describes its design and implementation.
\Cref{ch:Evaluation} then evaluates the usefulness of the tool and the resulting \glspl{schema}.
Finally, \cref{ch:Conclusion} summarizes the results and concludes the main text of the thesis.
These are followed by a \nameref{bibliography} listing sources,
a \nameref{main} with short definitions of most acronyms and terms used in this thesis,
and an \nameref{chap:appendix} with some ancillary content that does not belong in the main text.

\doclicenseLongText
Its source code is available in the repository at
\url{https://github.com/lucaswerkmeister/master-thesis}.
