%% LaTeX2e class for student theses
%% sections/content.tex
%% 
%% Karlsruhe Institute of Technology
%% Institute for Program Structures and Data Organization
%% Chair for Software Design and Quality (SDQ)
%%
%% Dr.-Ing. Erik Burger
%% burger@kit.edu
%%
%% Version 1.3.3, 2018-04-17

\chapter{Introduction}
\label{ch:Introduction}

As Wikidata,
the free knowledge base in the Wikimedia movement,
continues to grow in volume and scope and is used by more and more other parties, % TODO “other parties” is weird; TODO citation for growth?
its data quality has been identified as one of the most important areas of development in the future \cite{wdcon2017-sotp}: % TODO \cite before colon looks ugly
in order for Wikidata to be useful,
its data must be trustworthy and available in a consistent format.
Unchecked vandalism discourages data reuse,
while inconsistent data models make it significantly more difficult or even impossible.

To combat these problems,
several quality control mechanisms are used on Wikidata.
Recently, editors have begun exploring the use of \acrlong{shex}
as another quality control mechanism to use.
Compared to the more established, Wikidata-specific quality constraints system,
\acrlong{shex} are more powerful and expressive,
and are also not specific to Wikidata alone.
However, schemas for \acrlong{shex} are tedious to write by hand.

Automatically inferring schemas from Wikidata items
promises to simplify the schema authoring process:
instead of manually putting together the schema,
describing shapes for different classes of items,
one simply selects a set of items,
and a schema is automatically generated based on the data about these items.
If the selected items have been carefully edited
to conform to a pre-existing schema,
perhaps described informally or only present in the minds of the editors,
then the result may be a formalization of that schema;
alternatively, applying the same process to a less curated set of input items
may result in a concise summary of the current schema of those items % TODO “concise” is not really the right word…
and possibly even demonstrate problems in the input data.

% TODO probably mention WikiProject ShEx?

The rest of this thesis % TODO a) “the rest” or just “this thesis”? b) “this thesis”? “this work”?
is organized as follows.
\Cref{ch:Background} explains concepts that are required
to understand the rest of the work. % TODO the rest of the thesis?
\Cref{ch:RDF2Graph+Wikidata} describes general updates for RDF2Graph
as well as changes that were made to add Wikidata support to it.
\Cref{ch:wdsi} introduces the Wikidata Shape Expressions Inference Tool
and describes its design and implementation.
\Cref{ch:Evaluation} then evaluates the usefulness of the tool and the resulting schemas.
Finally, \cref{ch:Conclusion} summarizes the results and concludes the thesis.
